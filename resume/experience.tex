\cvsection{Experience}
\begin{cventries}

    \cventry
    {Junior Agile Coach}
    {Müstakil Girişimciler Derneği}
    {Istanbul}
    {September 2019 - Present}
    {
        \begin{cvitems}
            \item {They want to be a digital association. They want to move online all works! I've talked about agile and I'm helping them with all this agile transformation.}
        \end{cvitems}
    }
    
  \cventry
    {Full Stack Engineer}
    {Freelancer}
    {}
    {June 2014 - Present}
    {
      \begin{cvitems}
        \item {Pek çok web projelerinde framework kullanmadan Php ile geliştirme yaptım.}
        \item {Microservis yaklaşımı ve Linux Container teknolojisinin yükseleşiyle birlikte Php uzaklaşıp RESTFull mimariye geçtim. Front-End için React, Back-End için Golang, Node.js ve Rust ile geliştirme yapmaktayım.}
      \end{cvitems}
    }
    
  \cventry
    {Co-Founder}
    {10POINTS (Start-up)}
    {İstanbul}
    {September 2017 - August 2018}
    {
      \begin{cvitems}
        \item {Çay ve kahve piyasasında gördüğümüz eksikliği fırsata çevirmek için projeyi hazırladık.}
        \item {Bu konu üzerinden yaklaşık bir sene çalıştık ve proje haline getirdik.}
      \end{cvitems}
    }
    
  \cventry
    {Full Stack Engineer Intern at Kentkart}
    {Kentkart (Intern)}
    {Kocaeli}
    {June 2018 - July 2018}
    {
      \begin{cvitems}
        \item {React.js, Node.js ve Php ile geliştirilmeye başlayan muhasebe raporlarını grafiklerle sunacak web uygulamasını geliştirdim.}
      \end{cvitems}
    }
    
  \cventry
    {Full Stack Engineer and DevOps Engineer at Kentkart}
    {Kentkart (Full-Time)}
    {Kocaeli}
    {July 2018 - March 2019}
    {
      \begin{cvitems}
        \item {Raporlama web React, Php, Nodejs, MySQL, MongoDB ve Redis kullanılarak şirket içindeki excel formatındaki veriler görselleştirilerek raporlama uygulaması geliştirildi.}
        \item {Hyper Casual tardaki KenKen adlı oyun Unity kullanılarak geliştirildi. Backend kısmında Nodejs, GraphQL ve PostgreSQL kullanıldı.}
        \item {Jenkins ile CI/CD pipeline oluşturulup. Gitlab master push edildiği zaman pipeline tetikleniyor ve CI/CD süreçleri sırasında sorun olduğunda Slack üzerinden bilgilendiriliyorduk. Tüm süreç otomotize şekilde işliyordu.}
        \item {AWS'de oluşturulan instance maliyetler nedeniyle Scaleway'e taşıdım. Tüm trafik Haproxy üzerinden geçerek private network'de dağıtılıyordu. Yazdığım BASH script ile domainimiz için Letsencrypt'den ücretsize 3 aylık TLS ediniyordu.}
        \item {Veritabanlarını cloud private networkte dışarıdan izole bir biçimde saklıyordum.}
        \item {Version control için GitLab jenkins ile entegre bir şekilde çalışıyordu.}
      \end{cvitems}
    }
    
\end{cventries}
